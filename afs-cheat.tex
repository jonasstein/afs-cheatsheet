\documentclass[a4paper,11pt,notumble]{leaflet}
\usepackage[utf8]{inputenc}
\usepackage[ngerman]{babel}
%\usepackage{setspace}
\usepackage{hyperref}


\usepackage{graphicx}
\graphicspath{{gfx/}}

\newcommand{\Key}[2]{{\bf {\tt #1}}\quad {#2}\\[2mm]}
\newcommand{\Keylong}[3]{{\bf {\tt #1}}\quad {#2} \quad  {\it #3} \\[2mm]}
\newcommand{\Icon}[1]{\includegraphics[width=7mm]{#1}}


\begin{document} 
\hypersetup{
  pdftitle={AFS Kurzreferenz},  
  pdfsubject={Kurzreferenz mit Tipps zum Andrew File System},
  pdfauthor={Jonas Stein, et al.}
  pdfnewwindow=true
}

{\Huge AFS Kurzreferenz}\\
{\small PDF erstellt am \today}



\section*{Dateirechte ACL}
Die Rechte werden per ACL (= Access Control List) für ein Verzeichnis gesetzt und gelten für alle Dateien im Ordner.

\Keylong{r}{read} {Datei darf gelesen werden}
\Keylong{w}{write}{Datei darf verändert werden}
\Keylong{l}{lookup}{Datei darf gelistet werden}        
\Keylong{i}{insert}{weitere Dateien dürfen hinzugefügt werden}        
\Keylong{d}{delete}{Dateien dürfen gelöscht werden}
\Keylong{k}{lock}{Dateien dürfen gesperrt  werden}
\Keylong{a}{administer}{ACL des Verzeichnisses dürfen geändert werden}





%\Icon{pastetags.png} \Key{Strg + Shift + v}{Tags des kopierten Objekts auf anderes Objekt übertragen}
%e\\[-48pt]

\section*{Rechte abfragen}
\Keylong{fs listacl}{zeigt die ACL vom aktuellen Verzeichnis. Pfad kann auch angegeben werden}{fs listacl -path /afs/project}


\section*{Rechte setzen}
\Keylong{fs setacl -clear}{löscht die ACL vom aktuellen Verzeichnis}{}
\Keylong{fs setacl}{setzt die ACL vom aktuellen Verzeichnis. Pfad kann auch angegeben werden}{fs setacl -path /afs/project}

\section*{wichtige Dateien}
\Keylong{/usr/local/etc/CellServDB}{}{}
\Keylong{/usr/local/etc/ThisCell}{}{}
\Keylong{/usr/local/etc/cacheinfo}{}{}


\newpage

\section*{Informationsquellen}
http://docs.openafs.org/Reference/


% \textbf{Dank für das Mitwirken an:} 
\end{document}

%%% Local Variables: 
%%% mode: latex
%%% TeX-master: t
%%% End: 
